\documentclass{article}
\usepackage{booktabs}
\usepackage{hyperref}
\usepackage{graphicx}
% \usepackage
% [
%         a4paper,% other options: a3paper, a5paper, etc
%         left=1.5cm,
%         right=1.5cm,
%         top=2cm,
%         bottom=3cm,
%         % use vmargin=2cm to make vertical margins equal to 2cm.
%         % us  hmargin=3cm to make horizontal margins equal to 3cm.
%         % use margin=3cm to make all margins  equal to 3cm.
% ]
% {geometry}
\newcommand{\Tau}{\mathrm{T}}
\usepackage[utf8]{inputenc}
\usepackage[polish]{babel} 
\usepackage[T1]{fontenc}		% For font coding
\usepackage{polski}				% For word breaking dictionary
\usepackage{amsmath}
\newsavebox{\myproof}
\usepackage{amssymb}

\newcommand*{\field}[1]{\mathbb{#1}}%

\title{WSTĘP DO MATEMATYKI FINANSOWEJ \\
LABORATORIA 3\\}
\author{Stanisław Pawlak}
%\date{November 2019}

\begin{document}
\maketitle
% \tableofcontents
% \newpage
% \section{Zadanie 1}
\section{Zadanie}
Udowodnij, że $\varphi \in \Phi$ wtedy i tylko wtedy, gdy $V_{t}(\varphi) = V_{0}(\varphi) +G_{t}(\varphi)$ dla $t \in \Tau$.
\newline


\textbf{Rozwiązanie:}
\begin{itemize}
\large
    \item [$\Leftarrow$]
    Weźmy różnicę $V(\varphi)$ podstawiając z założenia :
    $$V_{t+1}(\varphi) - V_{t}(\varphi) = \varphi_{t+1}(S_{t+1} - S_{t})$$
    
    Z definicji $V(\varphi)$:
    $$ V_{t+1}(\varphi) - V_{t}(\varphi) = \varphi_{t+1}S_{t+1} - \varphi_{t}S_{t} $$
    
    Porównując otrzymujemy: $\varphi_{t+1}S_{t} = \varphi_{t}S_{t}$. Oznacza to, że $\varphi \in \Phi$ dla $t \in \Tau$.
    

    \item [$\Rightarrow$]
    
   $$V_{t}(\varphi) =  \varphi_{t}S_{t} = \varphi_{0}S_{0} + \sum_{u=0}^{t-1} (\varphi_{u+1}S_{u+1} - \varphi_{u}S_{u})$$
    
    Podstawiając $\varphi_{u}S_{u} = \varphi_{u+1}S_{u}$
    $$V_{t}(\varphi) =   \varphi_{0}S_{0} + \sum_{u=0}^{t-1} (\varphi_{u+1}(S_{u+1} - S_{u})) =   V_{0}(\varphi) + G_{t}(\varphi)$$ 
\end{itemize}


\section{Zadanie}

Niech $C_{0}$oraz $P_{0}$ oznaczają ceny europejskich opcji kupna i sprzedaży z tą samą ceną wykonania K oraz tym samym terminem wykupu T. Udowodnij, że jeśli parytet kupna-sprzedaży
$$C_{0} - P_{0} = S_{0} - \frac{K}{1+ r}$$

nie jest spełniony, to na rynku istnieje arbitraż.

\textbf{Rozwiązanie:}
\begin{itemize}
    \item [a)] Przypadek pierwszy: $C_{0} - P_{0} > S_{0} - \frac{K}{1+ r}$. Wtedy: \\
    $$K + (1+r)(C_{0} - P_{0} - S_{0}) > 0$$
   
   Strategia arbitrażowa polega na kupnie akcji i opcji sprzedaży sprzedaniu opcji kupna.
   
   \item [b)] Przypadek drugi: $C_{0} - P_{0} < S_{0} - \frac{K}{1+ r}$. 
   
   Analogicznie. Przyjmujemy strategię przeciwną do opisanej powyżej.
\end{itemize}

\section{Zadanie}
Załóżmy, że na rynku istnieje strategia samofinansująca $\varphi \in \Phi$ taka, że: \\
$V_{0}(\varphi) = 0$ oraz $P(V_{t_{0}} (\varphi) \geq 0) = 1$ oraz $P(V_{t_{0}} (\varphi) > 0) > 0$,
gdzie $t_{0} \in {0, . . . , T - 1}$ jest ustalonym czasem. O tej strategii można myśleć jako o (lokalnej) strategii arbitrażowej w przedziale czasowym $[0, t_{0} ]$. Udowodnij, że na rynku istnieje strategia arbitrażowa w $[0, T ]$. \\

\textbf{Rozwiązanie:} \\
Strategia arbitrażowa w  $[0, T ]$ powstaje poprzez włożenie w chwili $t_{0+1}$ wszystkiego do banku, aż do chwili $T$.

W chwili T otrzymujemy:
 $P(V_{T} (\varphi) \geq 0) = 1$ oraz $P(V_{T} (\varphi) > 0) > 0$
Wraz z początkowym warunkiem: $V_{0}(\varphi) = 0$ spełnione są wszystkie warunki na strategię arbitrażową.


\section{Zadanie}

Udowodnij, że gdy istnieje strategia $\varphi$ spełniająca $V_{0} (\varphi) < 0$ oraz \\ $P(V_{T} (\varphi) \geq 0) = 1$, to na rynku istnieje arbitraż. \\

\textbf{Rozwiązanie:} \\
Istnienie strategii arbitrażowej wydaje się oczywiste, ponieważ w chwili 0 wartość portfela jest ujemna, zaś w chwili T: $V_{T}(\varphi) \geq 0$ z prawdopodobieństwem 1. Oznacza to, że w sposób pewny można otrzymać zysk o wartości $-V_{0}(\varphi)$.

\section{Zadanie}

Podaj przykład rynku bez arbitrażu, gdzie nie wszystkie wypłaty są osiągalne. Opisz dokładnie przestrzeń strategii samofinansujących $\Phi$ oraz przestrzeń wypłat osiągalnych. \\

\textbf{Rozwiązanie:} \\
Przykładem takiego rynku może być rynek analogiczny do omawianego w pkt. (5.11) wykładu: \\

Rynek jednookresowy, trzystanowy $(\Omega = \{\omega_{1},\omega_{2}, \omega_{3}\})$ z jedną akcją. $r = 0.1$ oraz $S_{0}^{1} = 20$, $S_{1}^{1}(\omega_{1}) = 11$, $S_{1}^{1}(\omega_{2}) = 22$, $S_{1}^{1}(\omega_{3}) = 33$.

Na rynku nie ma arbitrażu, rynek nie jest zupełny.
$S_{0} = (1, 20)^{T}$

Otrzymujemy dwa wektory niezależne: $ S_{1}^{0}(\omega) = (22, 22, 22)^{T}$, $S_{1}^{1}(\omega) = (11, 22, 33)^{T}$

$\Phi = \{\varphi_{0}: \varphi_{0}S_{0} = \varphi_{1}S_{0}\}$. Przestrzeń $\Phi$ zawiera wszystkie kombinacje liniowe wektorów $\varphi_{0} = (1, 0)$ oraz $\varphi_{0} = (0, 1)$, $\varphi = (a,b)$.

Przestrzeń wypłat osiągalnych określa: $a(22, 22, 22)^{T} + b(11, 22, 33)^{T}$.

\section{Zadanie} 
 Podaj przykład rynku bez arbitrażu, gdzie istnieje wiele strategii replikujacych daną wypłatę. \\
 
\textbf{Rozwiązanie:} \\
Rynek jednookresowy, dwustanowy $(\Omega = \{\omega_{1},\omega_{2}\})$ z dwoma akcjami. $r=0$ oraz:
$$S_{0}^{1} = 10, S_{1}^{1}(\omega_{1}) = 15, S_{1}^{1}(\omega_{2}) = 5,$$ 
$$S_{0}^{2} = 5, S_{1}^{2}(\omega_{1}) = 10, S_{1}^{2}(\omega_{2}) = 0,$$ 

Strategie $\varphi = (a, b, c)$ dla $a + 10b + 5c = 100$, $a,b,c \in \field{N}\cup \{0\}$ replikują tą samą wypłatę.
\end{document}